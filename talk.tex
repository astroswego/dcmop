\documentclass{beamer}
%
% Choose how your presentation looks.
%
% For more themes, color themes and font themes, see:
% http://deic.uab.es/~iblanes/beamer_gallery/index_by_theme.html
%
\mode<presentation>
{
  \usetheme{Warsaw}%Madrid}      % or try Darmstadt, Madrid, Warsaw, ...
  \usecolortheme{beaver}%seahorse % or try albatross, beaver, crane, ...
  \usefonttheme{default}  % or try serif, structurebold, ...
  \setbeamertemplate{navigation symbols}{}
  \setbeamertemplate{caption}[numbered]
} 

\usepackage[english]{babel}
\usepackage[utf8x]{inputenc}
\usepackage{adjustbox}
\usepackage{tabularx}

\usepackage{tikz}
\usepackage{pgfplots}
\pgfplotsset{compat=1.10}
%\usepgfplotslibrary{external} 
%\tikzexternalize
\newlength\figureheight
\newlength\figurewidth
\setlength\figureheight{0.65\textheight}
\setlength\figurewidth{\textwidth}

\title[Optimal Light Curve Modeling of Intrinsic Periodic Variables]{Optimal Light Curve Modeling of \\Intrinsic Periodic Variable Stars}
\author{Earl Bellinger}
\institute{} 
\date{\today}

\begin{document}
%\newcolumntype{C}[1]{>{\centering\arraybackslash}p{#1}}

%\frame{\maketitle}
\begin{frame}
  \titlepage
  \begin{center}
  %\begin{tabular}{ | C{2.6cm} | C{1.3cm} | C{1.3cm} | C{2.6cm} |} \hline
  \begin{tabular}{>{\centering\arraybackslash}p{2.6cm} 
                   >{\centering\arraybackslash}p{1.5cm} 
                   >{\centering\arraybackslash}p{1.5cm} 
                   >{\centering\arraybackslash}p{2.6cm}} 
    \includegraphics[height=1.8cm,keepaspectratio]{oswego.jpg} &
    \includegraphics[height=1.8cm,keepaspectratio]{iu.png} &
    \includegraphics[height=1.8cm,keepaspectratio]{iucaa.jpg} &
    \raisebox{.05\height}{\includegraphics[width=2.58cm,height=1.5cm]{nist.png}} 
\end{tabular}
\end{center}
\end{frame}

% Uncomment these lines for an automatically generated outline.
%\begin{frame}{Outline}
%  \tableofcontents
%\end{frame}

\section{Classical Light Curve Modeling}
\subsection{Motivation}
\begin{frame}{The Cepheid Period-Luminosity Relation}
\begin{enumerate}
\item Asteroseismology 
\item Stellar Evolution 
\item Cosmology 
\end{enumerate}
\end{frame}

\subsection{Fourier Analysis}
\begin{frame}{Fourier Analysis}
Let 
\begin{center}
\only<1>{
\begin{tabular}{c c r}
\textbf{Variable} & \textbf{Type} & \textbf{Meaning} \\ \hline \hline
$\boldsymbol \omega$ & vector & angular frequencies \\
$\mathbf k$ & vector & wavenumbers \\
$\mathbf A$ & tensor & amplitudes \\ 
$\boldsymbol \Phi$ & tensor & phases \\ \hline
\end{tabular}}
\only<2>{
\begin{tabular}{c c r}
\textbf{Variable} & \textbf{Type} & \textbf{Meaning} \\ \hline \hline
$\boldsymbol \omega$ & vector & angular frequencies \\
$\mathbf k$ & vector & wavenumbers \\
$\mathbf S$ & tensor & sine amplitudes \\ 
$\mathbf C$ & tensor & cosine amplitudes \\ \hline
\end{tabular}}
\end{center}
The magnitude $m$ at time $t$ is then
\only<1>{
$$m(t | \boldsymbol \omega, \mathbf A, \boldsymbol \Phi) = 
  \sum_{(k_1, \dots, k_{|\boldsymbol \omega|})}
    A_{\mathbf{k}} 
    \sin \left( 
      \mathbf{k} \boldsymbol \omega t + \Phi_{\mathbf{k}}
    \right)$$}
\only<2>{
$$m(t | \boldsymbol \omega, \mathbf S, \mathbf C) = 
  \sum_{(k_1, \dots, k_{|\boldsymbol \omega|})}
    S_{\mathbf{k}} 
    \sin \left( 
      \mathbf{k} \boldsymbol \omega t
    \right)
  + C_{\mathbf{k}} 
    \cos \left( 
      \mathbf{k} \boldsymbol \omega t
    \right)$$}
\onslide<2>{
$$\text{with } S_{\mathbf{k}} = A_{\mathbf{k}}\cos(\Phi_{\mathbf{k}})
  \text{ and } C_{\mathbf{k}} = A_{\mathbf{k}}\sin(\Phi_{\mathbf{k}})$$}
\end{frame}

\begin{frame}{Example of a Multiperiodic Oscillator}
\begin{figure}
\centering
\input{mpo.tikz}
\end{figure} 
\end{frame}

\begin{frame}{Noisy, Unevenly Spaced Samples}
\begin{figure}
\centering
\input{mpo-points.tikz}
\end{figure} 
\end{frame}

\subsection{Classical Light Curve Modeling}
%\begin{frame}{Linear Transformation}
%\begin{block}{Trigonometric Identity: Sine of Sum}
%Recall that 
%\begin{align*} A \sin(x + \Phi)
%  &= A \cos(\Phi) \sin(x) + A \sin(\Phi) \cos(x)
%\\&= S \sin(x) + C \cos(x) \end{align*}
%with $S = A\cos(\Phi)$ and $C = A\sin(\Phi)$.
%\end{block}
%So we may move toward linear parameters like so:
%$$m(t | \boldsymbol \omega, \mathbf S, \mathbf C) =
%    \sum_{(k_1, \dots, k_{|\boldsymbol \omega|})}
%      S_{\mathbf{k}} \sin \left( \mathbf{k} \boldsymbol{\omega} t \right) + %C_{\mathbf{k}} \cos \left( \mathbf{k} \boldsymbol{\omega} t \right) $$
%\end{frame}

\begin{frame}{Projection into Fourier Space}
\makebox[\textwidth][c]{\input{mpo-fourier.tikz}}
\end{frame}

\begin{frame}{Scaled by Observed Magnitudes}
\makebox[\textwidth][c]{\input{mpo-fourier-scaled.tikz}}
\end{frame}

\begin{frame}{Least-Squares Trigonometric Regression}%Building the Linear Equation Coefficient Matrix}
\begin{center}
\begin{tabular}{lcr}
\textbf{Variable} & \textbf{Size} & \textbf{Meaning} \\ \hline\hline
$k$ & 1 & maximum wavenumber \\
$\mathbf{t}, \mathbf{m}$ & $n$ & times and magnitudes \\
$\boldsymbol \omega$ & $p$ & angular frequencies \\
%$\mathbf S$, $\mathbf C$ & ${k^p\times k}$ & sine/cosine amplitudes \\ 
$\mathbf X$ & ${n \times (2k^p+1)}$ & coefficient matrix \\ \hline
\end{tabular}
\end{center}
%Now we can build a feature matrix of size ${n \times (2pq+1)}$ with 
Find $\boldsymbol \omega$ via periodogram, then compute $(\textsf{\textbf{X}}^\text{T} \textsf{\textbf{X}})^{-1} \textsf{\textbf{X}}^\text{T} \mathbf{m}$ with $\textbf{X} = $ \setlength{\arraycolsep}{4pt} $$ 
\hspace*{-3mm} \begin{bmatrix}
1 & \sin(t_1 \omega_1
  & \cos(t_1 \omega_1)
  & \sin(t_1 [\omega_1 + \omega_2]) & \cdots
  & \cos(t_1 \sum_j^{p} k\omega_j) \\[1.5ex]

1 & \sin(t_2 \omega_1)
  & \cos(t_2 \omega_1) 
  & \sin(t_2 [\omega_1 + \omega_2]) & \cdots
  & \cos(t_2 \sum_j^p k\omega_j) \\[1.5ex]

\vdots & \vdots 
       & \vdots 
       & \vdots & \ddots 
       & \vdots \\[1.5ex]

1 & \sin(t_n \omega_1)
  & \cos(t_n \omega_1) 
  & \sin(t_n [\omega_1 + \omega_2]) & \cdots
  & \cos(t_n \sum_j^p k\omega_j)
\end{bmatrix} $$
\end{frame}

\begin{frame}{Noisy, Unevenly Spaced Samples}
\begin{figure}
\centering
\input{mpo-points.tikz}
\end{figure} 
\end{frame}

\begin{frame}{Least Squares Fit}
\begin{figure}
\centering
\input{mpo-badfit.tikz}
\end{figure} 
\end{frame}

\section{Objectives and Constraints for Model Discovery}
\begin{frame}{Objectives and Constraints for Model Discovery}
We want
\begin{itemize}
\item To estimate parameters $(\boldsymbol{\hat \omega}, \mathbf {\hat A}, \boldsymbol{\hat \Phi})$ corresponding to a multiperiodic oscillator $\hat m$ that is best supported by the observed data ($\mathbf{t}, \mathbf{m}, \boldsymbol \epsilon$) \pause 

\item The find the simplest model; that is, the one with the fewest components needed to describe everything we witnessed 
$$\min \left\| \boldsymbol{\hat \omega} \right\|_1 \text{ and } \min \left\| \mathbf{\hat{A}} \right\|_1$$ \pause

\item To minimize loss between our model and the observations
$$\min \left\| \mathbf m - \hat m(\mathbf t | \boldsymbol{\hat \omega}, \mathbf {\hat A}, \boldsymbol{\hat \Phi})\right\|_2$$ \pause

\item And to be within the bounds of measurement error
$$|m_i - \hat m(t_i | \boldsymbol{\hat \omega}, \mathbf {\hat A}, \boldsymbol{\hat \Phi})| \leq \epsilon_i \text{ for all i}=1\ldots |\mathbf{m}| $$

\end{itemize}
\end{frame}

\section{An Optimization View of Photometric Modeling}
\begin{frame}{An Optimization View of Photometric Modeling}
\vspace{-3mm}
\begin{align*}
&&(\boldsymbol{\hat \omega}, \mathbf {\hat A}, \boldsymbol{\hat \Phi}) =
  \underset{(\boldsymbol \omega, \mathbf A, \boldsymbol \Phi)}{\arg\min} &
  \left(\left\|\boldsymbol \omega \right\|_1,
        \left\|\mathbf{A} \right\|_1,
        \left\| \mathbf{m} - \hat m(\mathbf{t}, \boldsymbol \omega, \mathbf A, \boldsymbol \Phi)\right\|_2
  \right) 
\\ &&\text{such that } &
  |m_i - \hat m(t_i | \boldsymbol \omega, \mathbf A, \boldsymbol \Phi)| \leq \epsilon_i 
    \; \text{for all } i=1\ldots |\mathbf{m}|
\\ &&\text{where } &
  \hat m(t | \boldsymbol \omega, \mathbf A, \boldsymbol \Phi) =
    \sum_{(k_1, \dots, k_{|\boldsymbol{\omega}|})}^{|\mathbf{A}|}
      A_{\mathbf{k}} \sin \left( \mathbf{k} \boldsymbol{\omega} t + \Phi_{\mathbf{k}} \right) 
\end{align*}
%\begin{overprint}
\begin{center}
\alt<2>{
  \begin{block}{\textbf{Properties of the Optimization}}
    \begin{tabular}{r c p{6.5cm}}
      \textbf{Multiobjective} &$\rightarrow$& Many criteria to optimize simultaneously\\
      \textbf{Constrained} &$\rightarrow$& Solution must satisfy specific conditions \\
      \textbf{Nonlinear} &$\rightarrow$& Constraints and objectives not directly proportional to inputs
    \end{tabular}
  \end{block}
}{
\begin{tabular}{c|c|c|c|c}
      \textbf{Variable} & \textbf{Type} & \textbf{Index} & \textbf{Meaning} & \textbf{Given?} \\ \hline\hline
      $\mathbf{t}$ & Vector & $i$ & Times of Observation & $\checkmark$ \\
      $\mathbf{m}$ & Vector & $i$ & Observed Magnitudes & $\checkmark$ \\
      $\boldsymbol \epsilon$ & Vector & $i$ & Observation Errors & $\checkmark$ \\
      $\boldsymbol \omega$ & Vector & $j$ & Angular Frequencies & $\times$ \\
      $\mathbf A$ & Tensor & $\mathbf{k}$ & Amplitudes & $\times$ \\
      $\boldsymbol \Phi$ & Tensor & $\mathbf{k}$ & Phases & $\times$ 
    \end{tabular}
}
\end{center}
%\end{overprint}
\end{frame}


%\subsection{Simplification with Independent Period Discovery}
%\begin{frame}{Simplification with Independent Period Discovery}
%Instead of estimating the frequencies $\boldsymbol \omega$ in the observed data simultaneously with finding the amplitude components, let's solve for these first independently and use that estimate as a constant in our optimization
%\begin{itemize}
%\item Least Squares Spectral Analysis (``Lomb-Scargle'') 
%\item Conditional entropy periodogram 
%\item CLEANest algorithm 
%\item etc
%\end{itemize}
%\end{frame}

\subsection{Scalarization, Linearization and Regularization}
\begin{frame}{Scalarization of the Objectives}
%We can reduce the complexity of our optimization problem by scalarizing our objectives with the Least Absolute Shrinkage and Selection Operator (LASSO):
We can apply the Least Absolute Shrinkage and Selection Operator (LASSO) to scalarize our objectives:
\begin{align*}(\boldsymbol{\hat \omega}, \mathbf {\hat A}, \boldsymbol{\hat \Phi})
  = \underset{(\boldsymbol \omega, \mathbf A, \boldsymbol \Phi)}{\arg\min}
  & \sum_i \left(m_i - \sum_{(k_1, \dots, k_{|\boldsymbol{\omega}|})} A_{\mathbf{k}} \sin \left(\mathbf{k} \boldsymbol{\omega} t_i + \Phi_{\mathbf{k}} \right) \right)^2
\\&+ \lambda \sum_{(k_1, \dots, k_{|\boldsymbol{\omega}|})} |A_{\mathbf{k}}|
\end{align*}
where the shrinkage parameter $\lambda$ can be chosen e.g.~via cross-validation. 
\vspace{3mm}
\begin{thebibliography}{1}
  \beamertemplatearticlebibitems
  \bibitem{lasso}
    Robert Tibshirani
    \newblock Regression Shrinkage and Selection via the Lasso
    \newblock {\em Journal of the Royal Statistical Society}, 1996
    \end{thebibliography}
\end{frame}

\subsection{Solving the Approximate Optimization Problem}
\begin{frame}{Solving the Approximate Optimization Problem}
\begin{block}{The LASSO Solution}
We can now use the coefficient matrix $\mathbf X$ from before to solve
$$\mathbf{\hat w} = \underset{\mathbf w}{\arg\min} \; \| \mathbf {m} - \mathbf {X} \mathbf {w} \|_2 + \lambda \| \mathbf{w} \|_1$$
using e.g.~coordinate descent (Wu, Tong, and Lange, 2008).
\end{block}
\vspace{5mm}
The entries of $\mathbf{\hat S}$ and $\mathbf{\hat C}$ can be found in successive columns of $\mathbf{\hat{w}}$, and finally $\mathbf{\hat A}$ and $\boldsymbol{\hat{\Phi}}$ can be restored using with
$$A_{\mathbf{k}} = \sqrt{S_{\mathbf{k}}^2 + C_{\mathbf{k}}^2} \; \text{ and } \; \Phi_{\mathbf{k}} = \tan^{-1} \left(\frac{C_{\mathbf{k}}}{S_{\mathbf{k}}}\right).$$
\end{frame}

\begin{frame}{Least Squares Fit}
\begin{figure}
\centering
\input{mpo-badfit.tikz}
\end{figure} 
\end{frame}

\begin{frame}{LASSO Fit}
\begin{figure}
\centering
\input{mpo-lasso.tikz}
\end{figure} 
\end{frame}

\subsection{To the Stars!}

\begin{frame}{Phase Plot of OGLE-LMC-CEP-0209}
\makebox[\textwidth][c]{\input{OGLE-LMC-CEP-0209-phased.tikz}}
\end{frame}

\begin{frame}{Least Squares Fit}
\makebox[\textwidth][c]{\input{OGLE-LMC-CEP-0209-ols.tikz}}
\end{frame}

\begin{frame}{LASSO}
\makebox[\textwidth][c]{\input{OGLE-LMC-CEP-0209-lasso.tikz}}
\end{frame}

\begin{frame}{Lasso vs.~Baart on the OGLE-III Catalog}
%\begin{table*}
  \begin{center}
  %\scalebox{0.63}{
  %\resizebox{\textwidth}{0.35\textheight}{
  \adjustbox{max height=\dimexpr\textheight-5cm\relax,
             max width=\textwidth}{
    \begin{tabular}{c c r c c l}\hline %{ c c c c c c } 
\textbf{Galaxy} & \textbf{Type} & \textbf{\# Stars} & \textbf{Median Lasso} $\mathbf{R^2}$ & \textbf{Median Baart} $\mathbf{R^2}$ & \textbf{Significance*} \\ \hline \hline
(all)&(all)&51018&0.8654 &0.8550 &$p< 0.001$\\
(all)&CEP&7448&0.9833 &0.9826 &$p= 0.00862 $\\
(all)&T2CEP&603&0.9133 &0.8949 &$p= 0.242 $\\
(all)&ACEP&89&0.9700 &0.9672 &$p= 0.0223 $\\
(all)&RRLYR&42878&0.8381 &0.8265 &$p< 0.001$\\
LMC&(all)&27273&0.7883 &0.7800 &$p< 0.001$\\
LMC&CEP&3070&0.9860 &0.9853 &$p= 0.0765 $\\
LMC&T2CEP&203&0.8653 &0.8540 &$p= 0.484 $\\
LMC&ACEP&83&0.9704 &0.9639 &$p= 0.00672 $\\
LMC&RRLYR&23917&0.7631 &0.7544 &$p< 0.001$\\
SMC&(all)&6619&0.9439 &0.9423 &$p= 0.00446 $\\
SMC&CEP&4353&0.9816 &0.9810 &$p= 0.0325 $\\
SMC&T2CEP&43&0.7827 &0.6882 &$p= 0.306 $\\
SMC&ACEP&6&0.9277 &0.9880 &$p= 0.953 $\\
SMC&RRLYR&2217&0.6540 &0.6455 &$p= 0.0375 $\\
BLG&(all)&17126&0.9582 &0.9532 &$p< 0.001$\\
BLG&CEP&25&0.9843 &0.9641 &$p= 0.254 $\\
BLG&T2CEP&357&0.9488 &0.9462 &$p= 0.275 $\\
BLG&RRLYR&16744&0.9583 &0.9533 &$p< 0.001$\\ \hline
\multicolumn{6}{l}{*P-values obtained by paired Mann-Whitney $U$ tests}
     \end{tabular}} 
  \end{center}
%\end{table*}
\end{frame}

\begin{frame}{OGLE-BLG-RRLYR-00319 (mode RRd)}
\makebox[\textwidth][c]{\input{OGLE-BLG-RRLYR-00319-phased-ts.tikz}}
\end{frame}

\begin{frame}{OGLE-LMC-CEP-2147 (mode 1O/2O/3O)}
\makebox[\textwidth][c]{\input{OGLE-LMC-CEP-2147-phased-ts.tikz}}
\end{frame}

%\begin{frame}{OGLE-LMC-CEP-2147 (mode 1O/2O/3O)}
%\makebox[\textwidth][c]{\input{OGLE-LMC-CEP-2147-phased-ts.tikz}}
%\end{frame}

%\begin{frame}{OGLE-LMC-CEP-0432 (mode F/1O)}
%\begin{figure}
%\input{OGLE-LMC-CEP-0432-photometry.tikz}
%\end{figure} 
%\end{frame}

%\begin{frame}{OGLE-LMC-CEP-2147}
%\begin{figure}
%\input{OGLE-LMC-CEP-2147-phased.tikz}
%\end{figure} 
%\end{frame}

%\begin{frame}{Leavitt's Law in the SMC and LMC}
%\begin{figure}
%\centering
%\includegraphics[height=0.8\textheight,keepaspectratio]{mean-light-pl.png}
%\end{figure}
%\end{frame}

%\begin{frame}{Mean Light Difference Between SMC and LMC}
%\begin{figure}
%\centering
%\includegraphics[height=0.8\textheight,keepaspectratio]{pl-difference.png}
%\end{figure}
%\end{frame}

%\begin{frame}{Multiphase Difference Between SMC and LMC}
%\begin{figure}
%\centering
%\includegraphics[height=0.8\textheight,keepaspectratio]{pl-difference-2.png}
%\end{figure}
%\end{frame}

\begin{frame}{Thanks for your attention!}
\begin{center} \textbf{Questions?} \end{center}
%\begin{block}{Acknowledgements}
%This is a joint work with Professor Shashi M. Kanbur and Daniel Wysocki at SUNY Oswego.
%\end{block}

  \begin{center}
  %\begin{tabular}{ | C{2.6cm} | C{1.3cm} | C{1.3cm} | C{2.6cm} |} \hline
  \begin{tabular}{>{\centering\arraybackslash}p{2.6cm} 
                   >{\centering\arraybackslash}p{1.5cm} 
                   >{\centering\arraybackslash}p{1.5cm} 
                   >{\centering\arraybackslash}p{2.6cm}} 
    \includegraphics[height=1.8cm,keepaspectratio]{oswego.jpg} &
    \includegraphics[height=1.8cm,keepaspectratio]{iu.png} &
    \includegraphics[height=1.8cm,keepaspectratio]{iucaa.jpg} &
    \raisebox{.05\height}{\includegraphics[width=2.58cm,height=1.5cm]{nist.png}} 
\end{tabular}
\end{center}
\end{frame}

\end{document}
